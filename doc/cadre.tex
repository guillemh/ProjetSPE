\documentclass[a4paper,10pt]{article}
\setlength{\textwidth}{17cm}
\setlength{\textheight}{25cm}
\setlength{\oddsidemargin}{-.7cm}
\setlength{\evensidemargin}{-.7cm}
\setlength{\topmargin}{-2cm}

\usepackage[french]{babel}
\usepackage[T1]{fontenc}
\usepackage[utf8]{inputenc}
\usepackage{amsmath,amssymb}
\usepackage{graphicx}
\usepackage{url}
\usepackage{xspace}
\usepackage{subfigure}
\usepackage{mathrsfs}

\title{Conduite de projet}

\author{Anne-Hermine Allain, \\ 
Jean-Philippe Bauchet, \\ 
Loïc Ciccone, \\ 
Héloïse Guillemaud}

\date{Jeudi 6 juin 2013}


\begin{document}

\maketitle

Introduction [...]


\section{Présentation du projet}

- projet orienté recherche
- objectifs techniques : implémenter un modèle SPH d'animation temps-réel de liquides
	- implémenter le modèle physique correspondant
	- d'un point de vue informatique, optimisation les calculs
	- réalisme et esthétique du résultat
- objectifs de délai : date de fin de projet, obtenir un exécutable fonctionnel avec des résultats acceptables


\section{Organisation et communication}

Organisation de l'équipe :
- 4 membres (équipe projet) et 2 encadrants (équipe étendue)
- parallélisation des tâches

Communication :
- mails de suivi, rencontres avec le prof (deux fois par semaine)
- séances de travail à 4 à l'école, le plus souvent (travail isolé réprouvé)
- dépôt Git
- éventuelle collaboration avec l'autre équipe


\section{Planning}

Planning :
- examens : du 24 au 29 pour la moitié de l'équipe, au 30 pour l'autre
- TD Sheme le 6 juin
- évaluation Sheme entre le 12 et le 14 juin
- rendez-vous réguliers avec les encadrants techniques
Description des étapes du projet :
- démarche : améliorations successives du modèle

\section{Ressources et contraintes}

Ressources :
- salle E301...
- documents fournis par le prof
- prérequis scolaires : cours de G3D, d'EDP
- contraintes : examens...

Risques :
- utilisation de QMake imprécise
- sous-parties du projet pas suffisamment testées car dépendantes d'autres modules
- risque d'etre coincé un moment donné, mais bonne disponibilité de nos encadrants qui sauront nous remettre sur la bonne voie


\end{document}
