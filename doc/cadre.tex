\documentclass[12pt]{article}
\setlength{\textwidth}{17cm}
\setlength{\textheight}{25cm}
\setlength{\oddsidemargin}{-.7cm}
\setlength{\evensidemargin}{-.7cm}
\setlength{\topmargin}{-2cm}

\usepackage[french]{babel}
\usepackage[T1]{fontenc}
\usepackage[utf8]{inputenc}
\usepackage{amsmath,amssymb}
\usepackage{graphicx}
\usepackage{url}
\usepackage{xspace}
\usepackage{subfigure}
\usepackage{mathrsfs}

% titre, auteur et date
\title{Grenoble INP - Ensimag \\ Gestion de projet de spécialité : Document de cadrage \\ Animation temps-réel de liquides}
\author{ALLAIN Anne-Hermine, BAUCHET Jean-Philippe, CICCONE Loïc, GUILLEMAUD Héloïse}
\date{\today}

\begin{document}

\maketitle

Dans ce document, nous présentons les méthodes de travail que nous allons utiliser dans le cadre de notre projet de spécialité, consistant en le développement d'un modèle d'animation de fluides. C'est pourquoi nous allons débuter ce document par une présentation du projet et des objectifs que nous nous sommes fixés. Nous aborderons ensuite les questions d'organisation de l'équipe, de planning, avant d'évoquer notre gestion du planning, mais aussi des ressources humaines et des compétences qui nous sont propres.

\section{Présentation du projet}

Ce projet de spécialité est davantage orienté vers la recherche que vers le génie logiciel. Il consiste en l'implémentation d'un modèle SPH {\em (Smoothed-Particle Hydrodynamics)} pour l'animation de fluides en temps réel. Ce modèle consiste à représenter un liquide ou un gaz sous la forme d'un ensemble de particules autonomes mais physiquement soumises à des forces et interagissant avec les particules qui lui sont voisines. La finalité de ce projet consiste en l'obtention d'une animation 3D la plus réaliste possible d'un fluide. La contrainte temps-réel impose un temps de calcul minimal pour un résultat quasi-instantané et favorisant l'interaction avec un tiers. Un exemple d'application possible de ce modèle est l'animation d'une cascade dans un jeu vidéo. \\

Techniquement, cet objectif s'articule autour de trois points : \\

\begin{itemize}
\item tout d'abord, l'implémentation du système et du modèle physique, ce qui passe par le calcul des forces et l'intégration de la seconde loi de Newton pour animer le fluide ;
\item d'un point de vue informatique, au vu de la complexité du calcul, et du nombre de particules considérées pour approximer un fluide, un objectif essentiel consiste en l'optimisation du calcul précédent, ce qui sous-entend par exemple l'utilisation de structures de données adéquates et performantes ;
\item enfin, nous devons veiller à obtenir un résultat réaliste, tant du point de vue physique qu'esthétique. Ceci consiste à construire un modèle partant de l'ensemble des particules et construisant la surface en mouvement. \\
\end{itemize}

Nous suivrons dans ce projet une démarche incrémentale dans notre amélioration du rendu et de la simulation, nous y reviendrons plus en détail ci-dessous. \\

Concernant le rendu de ce projet, le temps nous est insuffisant pour essayer plusieurs modèles physiques et implémenter toutes les optimisations possibles sur ce genre de problèmes. Notre objectif est donc, d'ici la date de fin du projet, d'obtenir un exécutable fonctionnel fournissant des résultats acceptables. Ces résultats incluent l'aspect visuel, l'aspect physique, mais aussi la recherche des meilleures performances. De plus, notre projet est par la suite susceptible d'être réutilisé par nos encadrants et des chercheurs pour des travaux ultérieurs, ce qui impose des contraintes de clarté, dans le code comme dans notre documentation. 

\section{Organisation et communication}

Notre équipe projet se compose de nous quatre, qui étudions tous en filière MMIS (Modélisation Mathématique, Images et Simulation). Deux encadrants, Pierre-Luc Manteaux et Marie-Paule Cani, sont nos principaux interlocuteurs en cas de difficultés sur le plan technique et forment l'équipe étendue. \\

Nos procédés de communication sont les suivants. Nous avons l'intention de travailler tous ensemble à l'Ensimag sur des tâches parallèles, éventuellement seuls ou en binômes. Pour des raisons d'efficacité et de compréhension des autres membres de l'équipe vis-à-vis des changements apportés au projet, nous avons tendance à réprouver le travail en solitaire, effectué le soir ou le week-end chez soi, quitte à rallonger nos horaires de présence à l'école en journée. \\

Pour synchroniser notre travail, nous utilisons un dépôt Git auquel les encadrants ont également accès, ce qui leur permet de contrôler notre travail et d'y revenir avec une plus grande connaissance lors de nos réunions de suivi. Concernant ces réunions justement, nous avons choisi d'adopter un rythme de deux suivis par semaine. Nous leur communiquons également par mail notre avancement et les éventuels problèmes rencontrés. \\

Par ailleurs, une autre équipe travaille sur un projet similaire. Une partie de notre travail est donc commune, ce qui rend possible un échange entre les deux équipes sur des points précis de l'algorithme.

\section{Planning}

Pour mener à bien ce projet nous décidons de mettre en place une approche incrémentale vis-à-vis des objectifs définis précédemment. Dans un premier temps, une étude approfondie de la littérature à été nécessaire pour assurer une compréhension poussée du modèle à implémenter. Au fur et à mesure du déroulement du projet, d'autres lectures seront nécessaires pour améliorer le modèle de base.\\

L'étude de la littérature a permis de mettre en exergue des étapes successives d'implémentation. Tout d'abord, un modèle physique ``basique'' pour permettre le mouvement d'un fluide modélisé à partir de particules. Cela comprend les forces qui agissent sur ces particules ainsi que la gestion des collisions. Une fois ce bloc de base fonctionnel, nous pourrons nous intéresser au rendu visuel du liquide. Ceci définit la deuxième grande partie du projet, notre équipe devra sans doute se diviser à partir de ce moment. Chaque sous-groupe pourra passer à l'amélioration de sa partie, notamment au niveau des performances du logiciel. Ces améliorations se baseront sur des publications, proposant de nouvelles méthodes d'implémentation.\\

Nous devrions donc obtenir successivement des animations de liquides de plus en plus rapides et réalistes. Il ne faut pas oublier non plus que pour le rendu final, nous devrons livrer un logiciel qui fonctionne, afin d'avoir un résultat concret à montrer.\\

À ne pas négliger non plus, les différentes étapes de suivi du projet. Nous prévoyons de rencontrer nos encadrants deux fois par semaine, nous devrons prendre du temps pour préparer ces rendez-vous, notamment avec des questions et un résumé de notre avancement depuis la dernière rencontre. \\

Planning :
- examens : du 24 au 29 pour la moitié de l'équipe, au 30 pour l'autre
- TD Sheme le 6 juin
- évaluation Sheme entre le 12 et le 14 juin
- rendez-vous réguliers avec les encadrants techniques
Description des étapes du projet :
- démarche : améliorations successives du modèle

\section{Ressources et contraintes}

Ressources :
- salle E301...
- documents fournis par le prof
- prérequis scolaires : cours de G3D, d'EDP
- contraintes : examens...

Risques :
- utilisation de QMake imprécise
- sous-parties du projet pas suffisamment testées car dépendantes d'autres modules
- risque d'etre coincé un moment donné, mais bonne disponibilité de nos encadrants qui sauront nous remettre sur la bonne voie


\end{document}
