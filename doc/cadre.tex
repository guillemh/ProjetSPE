\documentclass[a4paper,10pt]{article}
\setlength{\textwidth}{17cm}
\setlength{\textheight}{25cm}
\setlength{\oddsidemargin}{-.7cm}
\setlength{\evensidemargin}{-.7cm}
\setlength{\topmargin}{-2cm}

\usepackage[french]{babel}
\usepackage[T1]{fontenc}
\usepackage[utf8]{inputenc}
\usepackage{amsmath,amssymb}
\usepackage{graphicx}
\usepackage{url}
\usepackage{xspace}
\usepackage{subfigure}
\usepackage{mathrsfs}

% titre, auteur et date
\title{Grenoble INP - Ensimag \\ Gestion de projet de spécialité : Document de cadrage \\ Animation temps-réel de liquides}
\author{ALLAIN Anne-Hermine, BAUCHET Jean-Philippe, CICCONE Loïc, GUILLEMAUD Héloïse}
\date{\today}

\begin{document}

\maketitle

Introduction [...]


\section{Présentation du projet}

- projet orienté recherche
- objectifs techniques : implémenter un modèle SPH d'animation temps-réel de liquides
	- implémenter le modèle physique correspondant
	- d'un point de vue informatique, optimisation les calculs
	- réalisme et esthétique du résultat
- objectifs de délai : date de fin de projet, obtenir un exécutable fonctionnel avec des résultats acceptables


\section{Organisation et communication}

Organisation de l'équipe :
- 4 membres (équipe projet) et 2 encadrants (équipe étendue)
- parallélisation des tâches

Communication :
- mails de suivi, rencontres avec le prof (deux fois par semaine)
- séances de travail à 4 à l'école, le plus souvent (travail isolé réprouvé)
- dépôt Git
- éventuelle collaboration avec l'autre équipe


\section{Planning}

Pour mener à bien ce projet nous décidons de mettre en place une approche incrémentale vis-à-vis des objectifs définis précédemment. Dans un premier temps, une étude approfondie de la littérature à été nécessaire pour assurer une compréhension poussée du modèle à implémenter. Au fur et à mesure du déroulement du projet, d'autres lectures seront nécessaires pour améliorer le modèle de base.\\

L'étude de la littérature a permis de mettre en exergue des étapes successives d'implémentation. Tout d'abord, un modèle physique ``basique'' pour permettre le mouvement d'un fluide modélisé à partir de particules. Cela comprend les forces qui agissent sur ces particules ainsi que la gestion des collisions. Une fois ce bloc de base fonctionnel, nous pourrons nous intéresser au rendu visuel du liquide. Ceci définit la deuxième grande partie du projet, notre équipe devra sans doute se diviser à partir de ce moment. Chaque sous-groupe pourra passer à l'amélioration de sa partie, notamment au niveau des performances du logiciel. Ces améliorations se baseront sur des publications, proposant de nouvelles méthodes d'implémentation.\\

Nous devrions donc obtenir successivement des animations de liquides de plus en plus rapides et réalistes. Il ne faut pas oublier non plus que pour le rendu final, nous devrons livrer un logiciel qui fonctionne, afin d'avoir un résultat concret à montrer.\\

À ne pas négliger non plus, les différentes étapes de suivi du projet. Nous prévoyons de rencontrer nos encadrants deux fois par semaine, nous devrons prendre du temps pour préparer ces rendez-vous, notamment avec des questions et un résumé de notre avancement depuis la dernière rencontre. \\

Planning :
- examens : du 24 au 29 pour la moitié de l'équipe, au 30 pour l'autre
- TD Sheme le 6 juin
- évaluation Sheme entre le 12 et le 14 juin
- rendez-vous réguliers avec les encadrants techniques
Description des étapes du projet :
- démarche : améliorations successives du modèle

\section{Ressources et contraintes}

Ressources :
- salle E301...
- documents fournis par le prof
- prérequis scolaires : cours de G3D, d'EDP
- contraintes : examens...

Risques :
- utilisation de QMake imprécise
- sous-parties du projet pas suffisamment testées car dépendantes d'autres modules
- risque d'etre coincé un moment donné, mais bonne disponibilité de nos encadrants qui sauront nous remettre sur la bonne voie


\end{document}
